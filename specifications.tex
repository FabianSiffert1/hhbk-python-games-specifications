
\documentclass[12pt]{scrartcl}
\usepackage{ucs}
\usepackage[utf8x]{inputenc}
\usepackage[T1]{fontenc}
\usepackage[ngerman]{babel}
\usepackage{graphicx}
\usepackage{scrlayer-scrpage}

\clearpairofpagestyles
\chead{\headmark}
\cfoot*{\pagemark}
\setkomafont{pageheadfoot}{\sffamily}
\setkomafont{pagination}{}


\title{Pflichtenheft: MiniMax Algorithmus}
\author{Marvin Parchainski, Jascha Oelmann, Nils Rüttgers und Fabian Siffert}
\date{\today{},Düsseldorf}

\begin{document}

\begin{titlepage}
    \begin{center}
    %textbf bold/fett
    \textbf{
    \Huge{Pflichtenheft: MiniMax-Algorithmus}\\[15cm]
    }
    \Large{Marvin Parchainski, Jascha Oelmann, Nils Rüttgers und Fabian Siffert}\\[1cm]
    \large{Heinrich-Hertz-Berufskolleg\\{\today}}
    \end{center}
\end{titlepage}
%pagestyle empty -> Seitenzahl entfernt
\thispagestyle{empty}
\stepcounter{page}

 
\newpage

\tableofcontents
\thispagestyle{empty}

\newpage

\section{Einleitung}
\label{sec:einleitung}

MiniMax Algorithmus

\begin{itemize}
    \item Python
    \begin{enumerate}
        \item Games
        \item Pflichten
    \end{enumerate}
    \item Heft
\end{itemize}



Hier ein Verweis auf Abschnitt Nummer: \ref{sec:einleitung}

\section{Auftrag}
\subsection{Spielauswahl}
\subsubsection{Bauernschach}
\subsubsection{Dame}
\subsubsection{Tic-Tac-Toe}

\section{Bereits bestehende Systeme oder Produkte}
\section{Teams und Schnittstellen}
\section{Rahmenbedingungen}
\section{Technische Anforderungen}
\section{Problemanalyse}
\section{Qualität}
\section{Projektentwicklung}

		
% HOW TO: LATEX IMAGE
%\begin{center}
% \includegraphics[width=0.1\textwidth] {TUG-logo}
%\end{center}	
%Das Bild zeigt ein TUG-logo\footnote{Hier ein footnote Beispiel.}.


\end{document}