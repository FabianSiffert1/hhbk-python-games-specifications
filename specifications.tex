
\documentclass[14pt]{scrartcl}
\usepackage{ucs}
\usepackage[utf8x]{inputenc}
\usepackage[T1]{fontenc}
\usepackage[ngerman]{babel}
\usepackage{graphicx}
\usepackage{scrlayer-scrpage}
\usepackage{setspace} 

\clearpairofpagestyles
\chead{\headmark}
\cfoot*{\pagemark}
\setkomafont{pageheadfoot}{\sffamily}
\setkomafont{pagination}{}


\title{Pflichtenheft: MiniMax Algorithmus}
\author{Marvin Parchainski, Jascha Oelmann, Nils Rüttgers und Fabian Siffert}
\date{\today{},Düsseldorf}

\begin{document}
\onehalfspacing

\begin{titlepage}
    \begin{center}
    %textbf bold/fett
    \textbf{
    \Huge{Pflichtenheft: MiniMax-Algorithmus}\\[15cm]
    }
    \Large{Marvin Parchainski, Jascha Oelmann, Nils Rüttgers und Fabian Siffert}\\[1cm]
    \large{Heinrich-Hertz-Berufskolleg\\{\today}}
    \end{center}
\end{titlepage}
%pagestyle empty -> Seitenzahl entfernt
\thispagestyle{empty}
\stepcounter{page}

 
\newpage

\tableofcontents
\thispagestyle{empty}

\newpage
\setcounter{page}{1}
\section{Zielbestimmung}
\label{sec:zielbestimmung}

Die Computerspiele-Entwickler HHBK Tendo Research Centers hat uns damit beauftragt, Forschung bezüglich des MiniMax-Algorithmus zu Erstellungbetreiben. Ziel ist die Programmierung eines Prototypen zweier Spiele in vereinfachter Form. \par 


\subsection{Musskriterien}

\begin{itemize}
    \item Für den Prototypen sollen mindestens zwei der Spiele als Python Anwendung entwickelt werden.
    \item Für die GUI (Graphical User Interface) soll das tkinter-Modul, guizero oder Pygame verwendet
    \item Für die optimale Spielstrategie soll der Minimax-Algorithmus mit variabler Suchtiefe und
    entsprechender Bewertungsfunktion implementiert werden.
    \item Verwendung einer SQLite-Datenbank zur Verwaltung der Benutzerdaten und Spielstände.
    \item Anzeige der jeweiligen Spielregeln.
\end{itemize}


\subsection{Wunschkritierien}

\begin{itemize}
    \item Verwendung von Alpha-Beta-Pruning zur Optimierung des Minimax-Algorithmus.
    \item Möglichkeit zur Einbindung anderer KI’s (Vereinbarung einer Schnittstelle)
\end{itemize}


\subsection{Abgrenzungskriterien}

\begin{itemize}
    \item Das Spiel wird von einem Spieler am PC gespielt. Netzwerkfähigkeit ist nicht gefordert!
\end{itemize}

%MiniMax Algorithmus
%\begin{itemize}
%    \item Python
%    \begin{enumerate}
%        \item Games
%        \item Pflichten
%    \end{enumerate}
%    \item Heft
%\end{itemize}
%Hier ein Verweis auf Abschnitt Nummer: \ref{sec:zielbestimmung}

%\section{Auftrag}


%\subsection{Spielauswahl}
%\subsubsection{Bauernschach}
%\subsubsection{Dame}
%\subsubsection{Tic-Tac-Toe}

\newpage
\section{Teams und Schnittstellen}

Um eine gemeinsame Basis zu schaffen arbeitet unser Team zunächst als Einheit. Wir diskutieren, wie wir die Anforderungen des Lastenhefts des HHBK Tendo Research Centers am sinnvollsten erfüllen können. \par 

Auch die Programmierung beginnen wir zunächst zusammen und arbeiten nach dem Driver-System der Scrum-Arbeitsweise. 
Wichtig hierbei ist es, das die Spiele auf der selben Grundstruktur basieren. \par

Erst für die Programmierung der beiden individuellen Spiele teilen wir uns dann in zwei Teams auf. \par 

\section{Einsatzbereich}

Dieses Projekt ist ein Protoyp, somit ist eines der Ziele ein Proof-of-Concept. Hierbei geht es um das Testen der 
eigentlichen Spielestrategie, sowie um das Ausloten der vom Kunden gewünschten Funktionen \par

Sollte das Produkt sich bewähren, so wäre es im Freizeitbereich angesiedelt. \par

Die Zielgruppe sind hautpsächlich Strategiespiel-Interessierte Menschen im Alter von 12-99 Jahren. Wir setzen Basiskentnisse
bzgl. Computerbedienung und Computerspielen voraus. Die Spielregeln der beiden Spiele kann man sich über die GUI anzeigen lassen und sind somit
keine Voraussetzung.\par


\section{Produktübersicht}


Ein wichtiger Teil der Spiele ist das Verhalten der Künstlichen Intelligenz (KI). Dieses basiert auf dem MiniMax Algorithmus.



\section{Technische Anforderungen / Funktionen}

Wichtig ist die sinnvolle Integration eines modularen Fenstermanagement-Systems. \par

\section{Leistungen}

\section{Qualitätsanforderungen}

\section{Benutzeroberfläche}



\section{Sonstige Anforderungen}
\section{Technisches Umfeld}

Die technischen Anforderungen an die Soft- und Hardwaresysteme der Endgeräte sind gering.\par
Durch die Wahl des tkinter-Moduls haben wir sichergestellt, dass eine große Kompabilität zu verschiedenen Betriebssystemen gegeben ist. Die einzige Vorraussetzung ist
somit die Installation einer aktuellen Python-Version.\par 

\newpage
\section{Gliederung}

\begin{itemize}
    \item MiniMax Algorithmus
    \begin{enumerate}
        \item Erstellung des Pflichtenhefts in LaTEX
        \item Erstellung einer modularen GUI
        \begin{enumerate}
            \item Erstellen der verschiedenen Screens. Beispielsweise: Startbildschirm
            \item Erstellen der Oberfläche der Scoreboards
            \item Erstellen der Oberfläche der Spieleregeln
        \end{enumerate}
        \item Testen der GUI
        \item Erstellung beider Spiele
        \begin{enumerate}
            \item Implementation der Funktionen der Spieler
            \item Implementation des MiniMax Algorithmus
            \item Implementation der Künstlichen Intelligenz
            \item Implementation des Scoreboards \& einer entsprechenden Datenbank
            \item Implementation eines Speichersystems
            \item (Optional) Implementation von Alpha-Beta-Pruning
            \item (Optional) Implementation einer KI-Schnittstelle
            \item (Optional) Hilfefunktion: Einbledeoption für einen guten nächsten Zug
            \item (Optional) Implementation verschiedener Schwierigkeitsstufen
        \end{enumerate}
        \item Testen der Spiele
    \end{enumerate}
\end{itemize}

\newpage
\section{Ergänzungen}

\newpage
\section{Tests}
		
% HOW TO: LATEX IMAGE
%\begin{center}
% \includegraphics[width=0.1\textwidth] {TUG-logo}
%\end{center}	
%Das Bild zeigt ein TUG-logo\footnote{Hier ein footnote Beispiel.}.


\end{document}